\documentclass{gamadays}

% ------------------------------------------
% TITRE
% ------------------------------------------

\title{\textbf{An Agent-Based Model for Preemptive Evacuation Decisions During Typhoon}}

% ------------------------------------------
% AUTEUR(S)
% ------------------------------------------

% 1 auteur
% \author{L. Author \\
%  Employer, laboratory acronym}
% \date{email}
 
% plusieurs auteurs 
\author{R.C. Rodrigueza\up{1,2}, K.Chapuis\up{3}, MRJ.E. Estuar\up{2}\\[6pt]
\up{1} Sorsogon State University - Bulan Campus\\
\up{2} Ateneo de Manila University, ACCCRe\\
\up{3} Employer 3, laboratory acronym 3\\
}

\date{rcrodrigueza@gmail.com, kevinchapuis@gmail.com, restuar@ateneo.edu}

\begin{document}

\maketitle


\begin{keywords}
Typhoon, Preemptive Evacuation Decision, Agent-Based Model
\end{keywords}

\begin{abstract}
% (Obligatory.) A 500 words maximum abstract (28 lines). In any case, header (title, authors), keywords, abstract and additional material link must fit in 1 page. Please, do not change layout.\\

Natural disasters continue to cause tremendous damage to humans' lives and properties. The Philippines, due to its geographic location, is considered a natural disaster-prone country experiencing an average of 20 tropical cyclones annually. This condition necessitates the need to study what can be done to mitigate the effects of weather-related disaster. Understanding what factors significantly affects decision making during crucial evacuation stages could help in making decisions on how to prepare for disasters, how to act appropriately and strategically respond during and after a calamity. In this work, an agent-based model of human behavior during typhoon evacuation is presented. In the model, civilians are represented by households and their evacuation decisions were based from their calculated perceived risk. Also, rescuer and shelter manager agents were included as facilitators during the preemptive evacuation process. National and municipal census data was employed for the model, particularly for the characteristics of household agents. Further, geospatial data of a village in a typhoon-susceptible municipality was used to represent the environment. In the model, household agents are placed randomly on buildings while rescuers roam the area. Shelter managers are in designated evacuation shelters and are stationary. Additionally, conceptualized formulas are provided to attain the individual evacuation decision of household agents. The decision to evacuate or not to evacuate depends on the agent's perceived risk which also depends on three decision factors: characteristics of the decision maker (CDM); capacity related factors (CRF); and hazard related factors (HRF). Weights are assigned to each decision factor to assess their impact on the model outcome. Finally, the number of households who decided to evacuate or opted to stay as influenced by the model’s decision factors were determined during simulations. Linear regression was used to determine significant predictors to determine evacuation decision. Sensitivity analysis shows that all parameters used in the model are significant in the evacuation decision of household agents. Findings also show that total evacuation decision is more sensitive to weights assigned in capacity related factors, specifically type of house, floor level and past typhoon experience.
\vspace{0.5cm}

\end{abstract}

\begin{additionnalMaterial}
(Optional.) A link to additional material can be added here if necessary: images and videos (for demos), code, repository, website, etc.
\end{additionnalMaterial}


\end{document}

